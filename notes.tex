\section{Motivation}


When it comes to the prediction of stable chemical compounds,
in some sense the problem could be considered deterministic. What
we mean here is that if we have a measure of the total energy of a 
system before and after reaction, and that energy decreases we can
be reasonably certain the new system will form. The challenge then
is obtaining the total energy of a chemical system.

Because computation of the formation energy can be expensive in
terms of both time and computation, a surrogate is desirable. Machine
learning provided the material scientist a suite of methods for
the development of models.


\section{The architecture of the problem}


The provided data can be thought of as rows of pairs of elements and
there corresponding stability vector. The first approach might be to
envision a single model which could output the entir vector. There 
are two ways this could be designed if the goal is to use a single model.
First, a multilayer perceptron neural network can be comstructed with
10 layers, each of which is responsible for the prediction of a single
element of the stability vector. This way the resultant output is an
entire stability vector. The second single model approach capable of
producing the vector directly would be to use a one-vs-all multiclass
classification technique. We can picture any given possible stability
vector as a unit-vector* of 10-dimensional space, of which there are
10C2 total possible vectors. Each one of these vectors can be thought
of as a distinct class. Then the problem becomes one of assigning the
correct class label to a given pair of elements.

The issue with the above solutions is that they ignore our chemical
intuition. Primarily, they neglect the concept of stoichiometry. As
material scientist we know that for  abinary compound, elements tend
to combine in definite ratios. 

Consider approach 